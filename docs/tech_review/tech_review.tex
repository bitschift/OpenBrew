\documentclass[draftclsnofoot,onecolumn,letterpaper,10pt]{IEEEtran}
\pagestyle{empty}
\usepackage{geometry}
\geometry{textheight=9.5in, textwidth=7in}

\newcommand{\subparagraph}{}
\usepackage{titlesec}

\titleformat{\section}[block]{\bfseries\Large}{\thesection}{0.4em}{}
\titleformat{\subsection}[block]{\bfseries\large}{\thesubsection}{0.4em}{}
\titleformat{\subsubsection}[block]{\bfseries\normalsize}{\thesubsubsection}{0.4em}{}
\setlength{\parindent}{0pt}
\renewcommand{\thesection}{\arabic{section}}
\renewcommand{\thesubsection}{\thesection.\arabic{subsection}}
\renewcommand{\thesubsubsection}{\thesubsection.\arabic{subsubsection}}


\author{Connor Yates\\
\texttt{yatesco@oregonstate.edu\\}
\and
Aravind Parasurama\\
\texttt{parasura@oregonstate.edu\\}
\and
Cody Holliday\\
\texttt{hollidac@oregonstate.edu\\}}
\date{\today}
\title{brew.ai Client Requirements}
\begin{document}
\maketitle

\newpage
\tableofcontents
\newpage
\section{Overview}
This document provides a technical review of three broad topics within the brew.ai project: hardware, user interface, and machine learning.
Each broad topic will be discussed in its own section, preceded by an introduction by the section author.
After the introduction, a review of the potential technologies will be given.
Each broad topic will be divvied into three or more sub-topics, which will address specific technologies that will be combined to address the broad topic of the section.
Finally, each broad topic will have a recommendation about which specific technologies should be used.

\section{Controller Hardware}
\subsection{AVR Microcontrollers}
AVR microcontrollers provide an accessible and powerful solution for interfacing with multiple sensors and making multiple 
components work together.


\subsection{Thermocouple}
The thermocouple device is an essentially plug-and-play device for measuring immediate temperature, and can be attached to 
any GPIO on the AVR microcontroller. 
The advantage of a thermocouple over a thermistor is the lack of the tedious recalibration step needed everytime a thermistor 
needs to be replaced. 
It is worth noting the generally larger size, and increased cost of using a thermocouple.

\subsection{Peltier}
The peltier junction is a thermoelectric temperature control device that operates by creating a temperature difference on
either side of the device.
These devices range in price, and more expensive junctions can generate a greater temperature differential more efficiently, 
and also last longer.

\subsection{Carbon Sensor}
The carbon sensor will serve the purpose of measuring fermentation levels.

\subsection{Bluetooth}
Bluetooth will link the brewing hardware system with the Android front-end, as well as the learning back-end.

\section{User Interface}

\section{Machine Learning \\ -- \textbf{\textit{Connor Yates}}}
\subsection{Introduction}
While a vague title, this section investigaes one of the defining features of brew.ai.
In order to create an automated brewing system that not only controls the process in an automated fashion, but can learn from mistakes and improve upon the product, a method of artificial intelligence must be used.
There are several parts to the artificial intelligence setup for the brew.ai project.
This section will focus on three main aspects: learning algorithm, decision making structure, and preexisting implementations.
It is important to note that these aspects are not mutually exclusive. 
Decisions made in one section may effect the choices in another section.
However, there is still a large degree of freedom between each section, especially with regards to the preexisting software packages that are available.

\subsection{Learning Algorithm}
The class of learning algorithm used is a major choice when setting up the machine learning aspect of the project.

\subsubsection{Q-Learning}
\subsubsection{Bayesian Modeling with Model Averaging}
\subsubsection{Deep Learning}
Added not because I think it is a good choice for this type of problem, but rather because people like to talk about ``deep learning'', so this section aims to provide evidence as to why deep learning is not applicable to this type of problem.
\subsubsection{Genetic Algorithms}


\subsection{Decision Making Structure}
Some of these methods presented are heavily tied to a specific type of learning algorithm, eg Bayesian networks are mostly only of use when paired with Bayesian models.
However, the differences between these types of structures helps define the state space the learner will operate over.
For example, a neural network can be used as a continuous approximate of a Q-table in continuous reinforcement learning domains.
The decisions present in this section will shape how the problem domain is modeled.
\subsubsection{Neural Networks}
\subsubsection{State-Action Table}
\subsubsection{Bayesian Networks}

\subsection{Preexisting Implementations}
This section looks at pre-implemented libraries that are available for machine learning routines.
Considerations on memory usage, programming language API, and available functionality are the primary concerns for this sections.
\subsubsection{Theano}
\subsubsection{Keras/Tensorflow}
\subsubsection{FANN}
\subsubsection{Torch}



\end{document}
