\documentclass[draftclsnofoot,onecolumn,letterpaper,10pt]{IEEEtran}
\pagestyle{empty}
\usepackage{geometry}
\geometry{textheight=9.5in, textwidth=7in}

\newcommand{\subparagraph}{}
\usepackage{titlesec}

\titleformat{\section}[block]{\bfseries\Large}{\thesection}{0.4em}{}
\titleformat{\subsection}[block]{\bfseries\large}{\thesubsection}{0.4em}{}
\titleformat{\subsubsection}[block]{\bfseries\normalsize}{\thesubsubsection}{0.4em}{}
\setlength{\parindent}{0pt}
\renewcommand{\thesection}{\arabic{section}}
\renewcommand{\thesubsection}{\thesection.\arabic{subsection}}
\renewcommand{\thesubsubsection}{\thesubsection.\arabic{subsubsection}}


\author{Connor Yates\\
\texttt{yatesco@oregonstate.edu\\}
\and
Aravind Parasurama\\
\texttt{parasura@oregonstate.edu\\}
\and
Cody Holliday\\
\texttt{hollidac@oregonstate.edu\\}}
\date{\today}
\title{brew.ai Client Requirements}
\begin{document}
\maketitle

\newpage
\tableofcontents
\newpage
\section{Overview}
This document provides a technical review of three broad topics within the brew.ai project: hardware, user interface, and machine learning.
Each broad topic will be discussed in its own section, preceded by an introduction by the section author.
After the introduction, a review of the potential technologies will be given.
Each broad topic will be divvied into three or more sub-topics, which will address specific technologies that will be combined to address the broad topic of the section.
Finally, each broad topic will have a recommendation about which specific technologies should be used.

\section{Controller Hardware}

\section{User Interface}

\section{Machine Learning \\ -- \textbf{\textit{Connor Yates}}}
\subsection{Introduction}
While a vague title, this section investigates one of the defining features of brew.ai.
In order to create an automated brewing system that not only controls the process in an automated fashion, but can learn from mistakes and improve upon the product, a method of artificial intelligence must be used.
The artificial intelligence that must be imbued in the project has the specific goal of controlling the brewing process.
This is done by sending high level signals such as ``raise temperature'' or ``reduce stir rate'' to the hardware motor controller discussed in Aravind's section.
In order to make these decisions, a continual stream of data from the sensors is fed into the artificial intelligence module, which inform the decision.
Learning will be done in a ``online'' manner, since this allows improvements to the controller policy to happen in between batches \cite{RussellNorvig}.

There are several parts to the artificial intelligence setup for the brew.ai project.
This section will focus on three main aspects: learning algorithm, decision making structure, and preexisting implementations.
It is important to note that these aspects are not mutually exclusive. 
Decisions made in one section may effect the choices in another section.
However, there is still a large degree of freedom between each section, especially with regards to the preexisting software packages that are available.

\subsection{Learning Algorithm}
The class of learning algorithm used is a major choice when setting up the machine learning aspect of the project.
This will dictate how the controller policy will behave while we try to feed it data, which is the most complex part of the machine learning subsection.
\subsubsection{Q-Learning}
Q-learning is a traditional reinforcement learning technique where all possible states and actions are paired up, and a reward mapping between the current state and potential actions can be learned and exploited \cite{SuttonBarto}.
This method is based off a dynamic programming representation of learning, where knowledge from nearby states gets combined into the final value of the state-action pair.
This is important because it creates a solid method of temporal-difference learning \cite{SuttonBarto}, as it becomes possible to associate rewards to series of actions.
As more chains of actions are taken, it can become clear to the agent which actions are preferable in which states.
By learning the action-value function, which returns the most favorable reward at a given state, the agent learns to act optimally within the world.
\subsubsection{Bayesian Modeling with Model Averaging}
\subsubsection{Deep Learning}
Added not because I think it is a good choice for this type of problem, but rather because people like to talk about ``deep learning'', so this section aims to provide evidence as to why deep learning is not applicable to this type of problem.
\subsubsection{Genetic Algorithms}


\subsection{Decision Making Structure}
Some of these methods presented are heavily tied to a specific type of learning algorithm, eg Bayesian networks are mostly only of use when paired with Bayesian models.
However, the differences between these types of structures helps define the state space the learner will operate over.
For example, a neural network can be used as a continuous approximate of a Q-table in continuous reinforcement learning domains.
The decisions present in this section will shape how the problem domain is modeled.
\subsubsection{Neural Networks}
\subsubsection{State-Action Table}
\subsubsection{Bayesian Networks}

\subsection{Preexisting Implementations}
This section looks at pre-implemented libraries that are available for machine learning routines.
Considerations on memory usage, programming language API, and available functionality are the primary concerns for this sections.
\subsubsection{Theano}
\subsubsection{Keras/Tensorflow}
\subsubsection{FANN}
\subsubsection{Torch}

\subsection{Recommendation}


\end{document}
