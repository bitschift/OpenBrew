\documentclass[draftclsnofoot,onecolumn,letterpaper,10pt]{IEEEtran}
\pagestyle{empty}
\usepackage{geometry}
\geometry{textheight=9.5in, textwidth=7in}

\newcommand{\subparagraph}{}
\usepackage{titlesec}

\titleformat{\section}[block]{\bfseries\Large}{\thesection}{0.4em}{}
\titleformat{\subsection}[block]{\bfseries\large}{\thesubsection}{0.4em}{}
\titleformat{\subsubsection}[block]{\bfseries\normalsize}{\thesubsubsection}{0.4em}{}
\setlength{\parindent}{0pt}
\renewcommand{\thesection}{\arabic{section}}
\renewcommand{\thesubsection}{\thesection.\arabic{subsection}}
\renewcommand{\thesubsubsection}{\thesubsection.\arabic{subsubsection}}


\author{Connor Yates\\
\texttt{yatesco@oregonstate.edu\\}
\and
Aravind Parasurama\\
\texttt{parasura@oregonstate.edu\\}
\and
Cody Holliday\\
\texttt{hollidac@oregonstate.edu\\}}
\date{\today}
\title{brew.ai Client Requirements}
\begin{document}
\maketitle

\newpage
\tableofcontents
\newpage
\section{Overview}
This document provides a technical review of three broad topics within the brew.ai project: hardware, user interface, and machine learning.
Each broad topic will be discussed in its own section, preceded by an introduction by the section author.
After the introduction, a review of the potential technologies will be given.
Each broad topic will be divvied into three or more sub-topics, which will address specific technologies that will be combined to address the broad topic of the section.
Finally, each broad topic will have a recommendation about which specific technologies should be used.

\section{Controller Hardware}
\subsection{AVR Microcontrollers}
AVR microcontrollers provide an accessible and powerful solution for interfacing with multiple sensors and making multiple 
components work together.


\subsection{Thermocouple}
The thermocouple device is an essentially plug-and-play device for measuring immediate temperature, and can be attached to 
any GPIO on the AVR microcontroller. 
The advantage of a thermocouple over a thermistor is the lack of the tedious recalibration step needed everytime a thermistor 
needs to be replaced. 
It is worth noting the generally larger size, and increased cost of using a thermocouple.

\subsection{Peltier}
The peltier junction is a thermoelectric temperature control device that operates by creating a temperature difference on
either side of the device.
These devices range in price, and more expensive junctions can generate a greater temperature differential more efficiently, 
and also last longer.

\subsection{Carbon Sensor}
The carbon sensor will serve the purpose of measuring fermentation levels.

\subsection{Bluetooth}
Bluetooth will link the brewing hardware system with the Android front-end, as well as the learning back-end.

\section{User Interface}

\section{Machine Learning}


\end{document}
