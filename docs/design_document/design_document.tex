\documentclass[draftclsnofoot,onecolumn,letterpaper,10pt]{IEEEtran}
\pagestyle{empty}
\usepackage{geometry}
\geometry{textheight=9.5in, textwidth=7in}

\newcommand{\subparagraph}{}
\usepackage{titlesec}

\titleformat{\section}[block]{\bfseries\Large}{\thesection}{0.4em}{}
\titleformat{\subsection}[block]{\bfseries\large}{\thesubsection}{0.4em}{}
\titleformat{\subsubsection}[block]{\bfseries\normalsize}{\thesubsubsection}{0.4em}{}
\setlength{\parindent}{0pt}
\renewcommand{\thesection}{\arabic{section}}
\renewcommand{\thesubsection}{\thesection.\arabic{subsection}}
\renewcommand{\thesubsubsection}{\thesubsection.\arabic{subsubsection}}



\date{\today}

\title{brew.ai Design Document}

\begin{document}
{\huge\textbf{Senior Software Engineering Design Group 7}}
	\vspace{1cm}

{\Huge\textbf{brew.ai Design Document}}

\vspace{2cm}
\textbf{Connor Yates} yatesco@oregonstate.edu

\textbf{Aravind Parasurama} parasura@oregonstate.edu

\textbf{Cody Holliday} hollidac@oregonstate.edu

\vspace{2cm}
Sponsor

Dale McCauley, College of Business, Oregon State University

\vspace{0.5cm}
	Approved: \today{}

	Version: 1.0


\newpage
\begin{abstract}
	abstract goes here
\end{abstract}
\newpage
\tableofcontents
\newpage
\section{Introduction}
% Perhaps get rid of these subsection titles, and just combine it into one larger section with appropriate paragraph spacings.
% These talk about the Purpose, scope, etc, of this paper.
\subsection{Purpose}
\subsection{Scope}
\subsection{Context}
\subsection{Summary}

\section{Glossary}

\section{Stakeholders and Design Concerns}
% Here we list out the design concerns of stakeholders (users, etc...) that we address in this document.
% By laying out the design concerns now, we can address each in turn in the subsequent design sections.

\section{Design Viewpoints}
% Details each design point we choose for this project.
% The tech review had 3 technologies, with 3 choices/investigations for each tech.
% Each of the three technologies chosen should be designed in detail in its own "viewpoint".
% each viewpoint then has its own "view". There can be more than one of these.
% the views describe the specific implementations that each viewpoint covers. Refer to the doc for more info.

\subsection{Abstract Learning from Previous Trials} %connor's
\subsubsection{Learning Algorithm} % Replace with design viewpoint name.
% talk about what algorithm I chose.
% Any design considerations that this algorithm requires
% design of this piece, in the context of the project

\subsubsection{Decision Making Structures}

\subsubsection{Library Implementations}

\subsubsection{Overall Learning Structure}
% diagram showing the interaction between the three previous viewpoints

\subsection{Android-Based User Interface Design} %cody's
\subsubsection{Interface Device}

\subsubsection{UI Connection to controller}

\subsubsection{Dataflow to UI}

\subsubsection{Interface layout}


\subsection{Brewing Hardware and Electronic Controls} %aravind's section
The brewing hardware is expected to perform the commands received from the Android client, while also automatically managing
some basic regulation functions on its own. The hardware will collect data about the brewing process and relay that back to the
Android application.

\subsubsection{Temperature control implementations}
For temperature control, a peltier junction will be used, attached to a heatsink on one side. The temperature control module will
hang from the lid of the device, and touch the liquid so as to apply a temperature differential to the liquid itself. Peltier junctions 
operate with pulse-width modulation, and at the levels of current required, an H-BRIDGE will have to be utilized in conjunction with the 
peltier module.

\subsubsection{Data collection from brewing process}
For data collection, an array of sensors has been chosen:
\begin{enumerate}
	\item A thermocouple, for minimal calibration temperature data gathering
	\item A carbon dioxide sensor, for measuring current fermentation levels
	\item A digital hydrometer for measuring initial gravity
\end{enumerate}
\subsubsection{Hardware connection to the client}
Data will be transmitted to the front-end by means of bluetooth. A bluetooth module will be used that supports interfacing directly with AVR's
USART functionality. This bluetooth module should easily pair with the Android application and transfer data.

\subsubsection{Central control system}
To bring all of the hardware components together, we will be using the Teensy 2.0, powered by the ATmega32u4. This microcontroller allows for basic
routines to be performed, and has a variety of hardware functionality that will be extensively used in this project.

\section{Design Rationale}

\section{Summary}

% References
\bibliography{design_document}
\bibliographystyle{ieeetr}


\end{document}
