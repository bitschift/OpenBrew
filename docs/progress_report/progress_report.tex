\documentclass[draftclsnofoot,onecolumn,letterpaper,10pt]{IEEEtran}

\usepackage{geometry}
\geometry{textheight=9.5in, textwidth=7in}

\usepackage{float}
\usepackage{graphicx}
\usepackage{algorithm2e}
\usepackage{pgfgantt}
\usepackage{booktabs}

\newcommand{\subparagraph}{}
\usepackage{titlesec}

\begin{document}
\begin{center}
	{\huge\textbf{Senior Software Engineering Design Group 7}}
	\vspace{1cm}

	{\Huge\textbf{brew.ai Progress Report}}

	\vspace{2cm}
	\textbf{Connor Yates}\\yatesco@oregonstate.edu

	\textbf{Aravind Parasurama}\\parasura@oregonstate.edu

	\textbf{Cody Holliday}\\hollidac@oregonstate.edu

	\vspace{2cm}
	{\Large CS 461, Fall 2016}
	\vspace{1cm}
\end{center}

\begin{abstract}

\end{abstract}

\newpage
\tableofcontents
\newpage

\section{Purposes and Goals}% Connor
\subsection{Project Purpose}
Brew.ai is a hardware and software solution for automated brewing of mead or beer.
Currently, home brewing requires a lot of time, knowledge, and patience. 
As such, it is not accessible to amateurs, and brew.ai seeks to solve this problem. 
From amateurs to professional brewers, we want brew.ai to be useful by automating the brewing process and helping brewers make better tasting products. 

\subsection{Project Goals}
At the end of the project, we look to produce a physical device that contains the necessary electronics and software to control the brewing process.
The brew.ai device itself is a bucket lid that will fit over a brewing device and have various modules incorporated in it. 
The lid device will monitor and control temperature, send and receive commands/data to and from the Android application, and monitor fermentation status.
From the perspective of a user, setup will be essentially plug-and-play. 
No technical knowledge is needed beyond knowing how to pair a Bluetooth device an Android device, open an app, and put ingredients into a bucket.
brew.ai also will improve on its recipe over time by leveraging the power of reinforcement learning and the users own feedback on the product it creates.
A main goal of this project is creating a product with the ability to learn from each batch, and improve its performance.

\section{Current Project Status}% Cody?

\section{Problems Encountered}%Aravind?

\section{Retrospective}
The retrospective lays out a history of what has happened this development cycle, along with an analysis of our performance this term.
The history is presented first, in Section~\ref{sec:weekSummaries}, with summaries of our progress each week.
Analysis, presented in a table format in Section~\ref{sec:analysis}, with columns on positives, deltas, and actions.

\subsection{Week Summaries}\label{sec:weekSummaries}
This section outlines our progress from week to week.
Each week is numbered via its position in the academic term.
The summaries are presented in a weekly format, with each week presenting a summary of our performance.
This summary is compiled from our individual weekly updates into one single summary for the week. % Change this if need be, if the other format works better

\subsubsection{Weeks 1 and 2}
During this section of the class, projects were presented by their respective sponsors, and groups were chosen.
As we had already picked a project topic and found a sponsor, this section did not see much progress, apart from orienting ourselves with the class schedule.

\subsubsection{Week 3} % Connor
This week saw progress on administrative tasks, such as meeting with Dale, our sponsor, and becoming more familiar with his requirements for formal business proposals from our group.
We also created the GitHub repository for our project, and added all the necessary people as collaborators.
We are starting to acquire our hardware for the project, from a microcontroller to brewing containers. 
These parts we are acquiring at the moment are items we already have, or can build from items we have on hand.

\subsubsection{Week 4} % Connor
In an uneventful week, we made revisions to our problem statement which would be approved by Dale at a later date.
Initial brewing mead samples will start soon.
Additionally, we started to think about writing the next document, the requirements document.

\subsubsection{Week 5} % Connor
Aravind setup the GitHub hooks for waffle.io, so we can keep track of our issues in a coherent, visual manner.
We were occupied with work from other classes this week, but still completed a rough draft of the requirements document.
This document will be edited by Dale and us in the coming week, so we can make changes and produce a final document.
It would have been nice to have complete document to present to Dale this week, but unfortunately this was just not feasible with our busy schedules.

\subsubsection{Week 6}

\subsubsection{Week 7}

\subsubsection{Week 8}

\subsubsection{Week 9}

\subsubsection{Week 10}

\subsubsection{Week 11 (Finals Week)}


\subsection{Analysis of Performance}\label{sec:analysis}

\begin{center}
	\begin{tabular}{p{0.3\linewidth} p{0.3\linewidth} p{0.3\linewidth}}
		\toprule
		\textbf{Positives} & \textbf{Deltas} & \textbf{Actions} \\
		\midrule
		We have a good relationship with our sponsor Dale. It helps that he's a very relatable person who is easy to get along with. & Coordination on tackling tasks is still lackluster, and could be improved. & Utilizing our waffle.io page to keep meticulous records of issues and assignments, while tedious, would clarify who is doing what part of the assignments.\\
		We are flexible when scheduling meetings, and have not had issues coming up with specific meeting times to meet with Dale.& Documents are being prepared close to the deadlines, causing undue worry.& We organize and divide up our work well. But this generally only happens when we are all able to work on the document. This process could be done sooner, to allow for each person to work on the assignments at an easier pace.\\

		\bottomrule
	\end{tabular}
\end{center}


%\bibliography{progress_report}
%\bibliographystyle{ieeetr}

\section{Summary}

\end{document}
