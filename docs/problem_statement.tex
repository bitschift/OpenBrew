\documentclass[letterpaper,10pt]{article}
\pagestyle{empty}
\usepackage{geometry}
\geometry{textheight=9.5in, textwidth=7in}

\author{Connor Yates\\
\texttt{yatesco@oregonstate.edu}
\and
Aravind Parasurama\\
\texttt{parasura@oregonstate.edu}
\and
Cody Holliday\\
\texttt{hollidac@oregonstate.edu}}
\date{\today}
\title{OpenBrew Problem Statement}


\begin{document}
\maketitle

\begin{abstract}
	Home beer or mead brewing is not currently a process as simple and guided as brewing
	coffee. OpenBrew is a platform envisioning that reality; for potential amateur homebrewers 
	to utilize the power of highly optimized machine learning algorithms to brew good-tasting, 
	high quality beer or mead without needing to know the nuances of the chemistry involved. 
	The OpenBrew project encompasses micro-electronics hardware (sensors, microcontrollers, 
	and actuators) and statistical learning algorithms to control the hardware. Through these 
	components, a user will be able to easily incorporate a fully automated, intelligent home 
	brewing system into their existing setup, as well as use our product as a framework to 
	construct a new brewing system. The algorithms will control the hardware to influence the 
	brewing process, creating a better product with each batch.
\end{abstract}

\newpage

\section{Definition of Problem}

\section{Proposed Solution}
The solution we propose is twofold in nature.
The first piece is a hardware implementation, which is controlled by the second piece, 
the autonomous software.

\subsection{Hardware}
In order to create a solid and reliable automated home-brewing setup, our hardware 
implementation needs to be simple and modular to accommodate peoples' various setups.
We will construct our brewing setup with common, off-the-shelf components.
Additionally, a microcontroller will be paired with general purpose sensors will be 
introduced into the brewing setup to measure and control the brewing process.
These components should not be cost prohibitive for the average home-brewer.

\subsection{Software}
The most important software aspect we will be creating is the learned control policy 
to automate the brewing process. By providing a learning algorithm with the proper 
inputs at initialization, and throughout the brewing process, the computer can completely 
control the complex chemical process. This allows the computer to improve over time 
based on the feedback from the user. To simplify this from an engineering and user 
perspective, this suggests a three part software solution.

The first part would be a simple interface for the user to use so that they can 
interact with the control policy on a high level without needing to know concepts of 
machine learning. Secondly, there will need to be specific code for the microcontroller 
that can interface with a predetermined complex control policy. This interface would 
require the ability to translate inputs from the raw sensor data into a format for 
the policy to accept, and to take the  output of the policy and translate into specific 
actions. The last part is the machine learning and artificial intelligence aspect.
Here, we will need to construct a method of creating and training a control policy 
that can satisfy the following:

\begin{enumerate}
	\item Accept feedback from the user, and incorporate this feedback into the 
		policy for fine-tuning.
	\item Provide real-time control commands based on the continuous input that 
		can be used by the microcontroller.
	\item Does not require extensive amounts of data for bootstrapping the learning 
		of the policy.
\end{enumerate}

\subsection{Business Aspects}
Since this proposal is being supported by the Austin Entrepreneurship Program, our 
solution also needs to incorporate aspects of marketability. In order to shape our 
product into something people would be interested in, we will incorporate ideas 
from interviews with prospective customers into our design. We will also create 
business documents such as a lean canvas business model to help plan how this product 
will grow in the future.
\section{Performance Metrics}

\end{document}

