\section{Weekly Blog Posts}
\subsection{Aravind Parasurama}

\textbf{2016-10-14}

This week, we got the GitHub repository running, with the concerned parties being added as collaborators. Initial stages of design are being completed, as we now have an abstract as well as descriptions of the problem and our solution. The team also met with Dale, to get a better idea of what the Austin Enterpreneurship Program expects from this project. Continuing forward, the hardware implementation of the project needs to be designed and built so that we can start writing and optimizing software. All in all, the project is continuing well.

\textbf{2016-10-21}

This week, we received feedback on our project proposals. Over the next week, we will have to refine our proposal and have it further evaluated by Dale. We will begin making mead samples this weekend.

\textbf{2016-10-30}

This week I set up the web hooks for waffle.io, and we got a few documents submitted. Career week was busy, but I met some very interesting companies, and got to network with some very great people. Next week, we'll be updating our requirements doc in anticipation of the final due date this Friday. We'll need to set up a meeting with Dale to get an autograph and to update him with the project, and we'll need to start doing some market research as Dale requested. There's plenty of breweries in Corvallis, so we'll start there.

\textbf{2016-11-04}

This has been a busy week with interviews, the GRE, coursework and updating the requirements document. Next week should be just as busy.

\textbf{2016-11-11}

This is a late update. Last week was very busy, and we managed to get out documents polished for submission. This week will involve working on the tech review and design documents.

\textbf{2016-11-18}

Finished revising Tech Review. Thanksgiving holiday next week.

\textbf{2016-11-25}

Starting design document. Finishing it next week.

\textbf{2017-01-20}

Interviewing in Redmond, WA this week.

\textbf{2017-01-23}

Set up daily meetings with Dale and Frank. Working on the first hardware prototype.

\textbf{2017-01-27}

Ordering some parts for the hardware prototype, and writing a bunch of Teensy routines.

\textbf{2017-02-03}

Uneventful week, wrote more Teensy routines.

\textbf{2017-02-10}

Ordered more parts for the first hardware prototype, initial heating element mounting copmlete.

\textbf{2017-02-17}

Working on the midterm presentation this week.

\textbf{2017-02-24}

Practiced our pitches in class this week, continued work on prototype.

\textbf{2017-03-03}

Working on fabricating exterior pieces for the hardware prototype.

\textbf{2017-03-10}

Finalizing first hardware prototype this week, and beginning tests.

\textbf{2017-03-23}

Working on the final progress report.

\textbf{2017-04-07}

Working on setting up meeting times, and starting test brews.

\textbf{2017-04-14}

Resuming work from last term on second hardware prototype, updating some GitHub docs. Ordered parts for the second device.

\textbf{2017-04-21}

Continuing work on second hardware prototype. Arrived parts go in the device, some calibration stuff happens. Worked on poster.

\textbf{2017-04-28}

Working on poster and hardware prototype. Test brew was a partial success, fixed sensor issues and began another one.


\textbf{2017-05-05}

Toss the case design for the second device, replace with a repurposed old heat-stirrer. Configure Ph sensor to work.

\textbf{2017-05-12}

Work on the midterm report, prepare for expo. CAD design some extra box parts, finish up other work with the second device.

\textbf{2017-05-19}

Expo week. Turned in midterm report, prepared projects for expo, presented at expo.

\textbf{2017-05-26}

 If you were to redo the project from Fall term, what would you tell yourself?
Don't put a computer above large evaporating masses of liquid. Also, do the progress reports winter term.

 What's the biggest skill you've learned?
Designing and redesigning effective hardware prototypes to make brewing hardware.

 What skills do you see yourself using in the future?
Github and communication skills that I have learned over the course of this term will come in handy over the course of my career.

 What did you like about the project, and what did you not?
I liked learning about brewing, and designing a cool device using new technology that nobody had used for this purpose yet. There was continually new interesting things to learn, and I have had some fun converstaions at interviews because of this project. On the flip side, desigining and building physical hardware prototoypes to test brew with was a time-consuming challenge. There were many different factors that could affect the success of a brew, and the hardware had to conform to all of them while also featuring the functionality brew.ai needed. The process of making these prototypes got pretty frustrating at times.

 What did you learn from your teammates?
I learned cool stuff about how reinforcement learning works, and how to communicate effectively with other software engineers. I also picked up bits of knowledge on Linux and Android development and administration.

 If you were the client for this project, would you be satisfied with the work done?
Yes! The project successfully brews, learns, puts on an interesting show at Expos, and looks good while doing it.

 If your project were to be continued next year, what do you think needs to be working on?
Refined hardware prototypes, extended brew testing, and parallel brews using cloud technology are three key features that brew.ai could use development on if it were continued next year.

\subsection{Connor Yates}

\textbf{2016-10-14}

Apart from coming down with a nasty cold this week, things seem to be going well. Getting into the groove of things with the group has been good, and our hardware is beginning to be assembled. I feel like this was a good start for the project. Everything got done that needed to be completed, so it certainly could have gone worse!
Getting the GitHub wiki to work isn't the most graceful thing, but I'll keep working on it until it looks the way I want it to. Additionally, as hardware starts to roll in, we will be able to assemble some initial prototypes for our hardware implementation.

\textbf{2016-10-21}

This week didn't see much action. Rather, there were some minor tests with the GitHub wiki, and some planning for hardware implementations to try out this weekend. Additionally, we have started working on the revisions for the project description, and will get those approved by Dale next week. Next week will also see the writing of the second assignment,the requirements document.

\textbf{2016-10-30}

The main contribution for this week was the completion of the rough draft of the requirements document.
In the coming week, I look to refine the document within our group, and with our sponsor, as well as begin hardware tests with the brewing setups, and start researching and choosing electronics components to use.
(This update comes a bit late, mainly due to a busy schedule with midterms this week).

\textbf{2016-11-04}

Well, this week has been hellishly busy between classes and paper writing for my research lab.
The progress on the requirements document came up way to close to the deadline, since peoples schedules were out of synchronization.
I think a long meeting for our group is needed in the near future to rigorously structure how we will add issues, handle pull requests, and allocate our man-hours to effectively make steady progress on the project.

\textbf{2016-11-11}

Another late update...
I suppose this has taught me to set a specific time/day to do these, since relying on "after class on friday" doesn't work when there's no class Friday.
This week we divided up the tasks for the technical report. I'll be focusing on the learning aspect, and most likely dividing this section into the three subsections of learning algorithm, decision making architecture, and input structure/learning rate.
Entirely mutually exclusive, and choices in one section may need to be taken into consideration in other sections.

I also have pictures of our current brewing setup, which I will post up here once I figure out a good method of hosting the images.

\textbf{2016-11-18}

This week, my work on this class was focused on the tech review document. Due to unforeseen circumstances in other classes, the tech review needed extra time to be completed, hence its completion being on Tuesday morning.
This next week, I want to start in on the final paper for this class by co-opting the previous work we have into the final document.
Hopefully this paper will go more efficiently than the previous work.

Additionally, now with the tech review completed, I think the time for finalizing and acquiring the hardware has come. This will allow us to work on building the prototype over winter break and begin to record training data. The four weeks of break is too great of a data-gathering opportunity to pass up.

\textbf{2016-11-23}

I'm publishing this update a little early this week, in anticipation of the holiday.
This week so far hasn't seen any direct progress, aside from setting up some issues in the issue tracker.

The design document is the next deliverable, and as such should take the main priority for this project at the moment.
I plan on working on the document on the 26th, and making sure everyone contributes their parts by Tuesday. I want to have a reviewable draft to Dale by either Tuesday or Wednesday, to give us time to review the document and make any necessary edits.

Sticking to this plan is essential for maintaining progress in this class, as well as ensuring adequate progress in my other classes term projects. As such, I will stick to this plan to the best of my abilities.

\textbf{2016-12-01}

With the design document fully in progress, I decided to take a break to write this update.
We met with Dale this afternoon, and discusses the progress we made on the design document. There's plenty of time to finish the document before noon tomorrow, and I feel we are in a good place to get it done. The tasks we have left are:
- Finish filling out our respective sections. For me, this entails:
  - Writing up the designs of Q-learning algorithm, the state/action sets, and NN approximator.
  - Create diagrams showing the designs of the agent relationship, NN structure, and agent structure.
  - Discuss some of the testing structures I can use for the agent.
- Review and edit document.
- Test on the server
- Get signatures
- Turn in
- Sleep

And with all of that, our design document will be finished on time.

\textbf{2017-01-20}

This week we had our first few meetings of the term, both with Frank and Dale. Aravind was out of town for job interviews, but Cody and I were able to make meetings, and set up a weekly time (Monday's after our meeting with Dale) where we can meet as a group to work on the development of the project as a synchronous team.
After our meeting with Dale, we look to perform two interviews of potential customers in the coming week. This will serve as the start of the interview process, and will allow us to review and reform our interview process as we gather more information from different people. This is a bit of a slow start to the term, but I expect things to pick up quickly.

\textbf{2017-01-23}

AAH! The term started, and it seems I needed to get back into the swing of things more quickly.

This week saw some snow, and a really early class time for the main lecture. I really appreciate not having to go to it constantly this term...

This week we also worked on setting up regular meeting times with Dale and Frank, so that next week we will be able to start back in on the regular, in-person, communication.

\textbf{2017-01-27}

This week became lost time, as I became pretty sick during the later part of the week.
That caused several issues, mainly by building up the stack of work I need to finish apart from capstone (which unfortunatly pushed this update back).
As such, I am having trouble finding time to sit down and work on the AI side of the project.
We've completed 3 of our minimum 5 interviews so far, so we are making some progress on that end.
The interviews have been helpful, especially in seeing how most people interested in a product like this want decent temperature control.
This helps point toward a major area we can focus on in this prototype.

It looks like acquiring hardware is taking a while, but I'm not sure why thats the case.
If that trend holds, it could be disasterous for the completion of the project.

I would definitely say that the project is in a slump right now, and we need to make a concentrated effort to break out of it.

For the upcoming week, my main goal is to sit down by Thursday and finish a working implementation of the approximated Q-learning algorithm.

\textbf{2017-02-03}

Most of this week's progress has been delayed untill tomorrow (Saturday) as I had two midterms this week.
Even so, there is progress completing more interviews, which is always good, and concrete steps in what needs to happen next.
I am finishing up my week's interview tomorrow, as well as completing a majority of the first iteration of the AI development.
With the coming week, we need to continue our code development, as well as finish off some of the expo-related tasks, including a group leader to handle those.
I have good hopes for our week at the moment, but those hopes remain to be validated at our next meeting.

\textbf{2017-02-10}

This week my main focus has been on taking care of the AI, and the tasks as the team captain.
Unfortunatly, I had some work obligations come up in the middle of the week, causing some delays in finishing everything I wanted to finish. Luckily, I'll have time this weekend to finish up the AI, and I aim to do all of the initial OneNote work this evening.
Things are getting done on the project, which is reassuring, but finding time to sit down and work on the project in the middle of a busy term is quite difficult.
But, all I can do is keep pushing forward and put in as much work as I can, making sure to finish things as quickly as possible.

\textbf{2017-02-17}

This week was writing week! I finished up the OneNote assignment, organizing and doing the document review process.
The OneNote system seems interesting, but I honestly prefer Git and LaTeX. I also use Linux exclusivly, so using a Microsoft web app is a bit morally sickening...
This week I also spent a bunch of my time working on research projects outside of the class, working to help complete a journal paper. This helped cause the need to look for an extension on the presentation, as well as my entire Friday being dedicated to a grad school event. Quite the week, and the weekend will be just as busy... But we just gotta march on.

As of this writing, I've finished up my writing for the progress report, so now all that remains is to edit the sections together, and then it is done. The plan is that once this is finished, we can show it as the sufficient progress to get the extension on the presentation.

\textbf{2017-02-24}

This week was a bit uneventful, but the class this week for practicing pitching was actually really useful.
Seeing what Kevin had to say about our presentation was pretty eye opening, since I honestly hadn't considered people *not* being intrigued by a homebrewing device. But this will make the expo day go better, so its very useful.

I am making good progress on the AI code, and if I can find the time this weekend, I believe I can finish it.

\textbf{2017-03-03}

This week I set up Keras and Theano on the Raspberry Pi B+, the instructions for which I have replicated [here](https://github.com/bitschift/brew.ai/wiki/Setting-up-the-Pi).

I also signed up the group for expo, so that is out of the way. Luckily the next steps as the team captain do not seem to start until next term.
With the Q learning algorithm written, this weekend I will be working on implementing a simulator to generate a test set of training data for the AI.

I organized a group meeting earlier in the week, where we were able to meet and get some more work done. I'll continue doing this throughout the life of the project, since my code side is finishing up quickly. But I am getting frustrated with the rate of progress on this project, and I am worrying about the fate of the project.

\textbf{2017-03-10}

Whoopse, I had a test yesterday and I forgot to post this.
This last week I've been working to organize the group through our last development push for this term.
So the things we need to finish up are:
* Complete analysis on Q-learning agent
  * Test simulator
  * Analyze results and make into a short presentation
* Complete hardware integration
  * Lasercut/3D print new housing structures after redesign to avoid water vapour near the pi.
  * Get Pi talking to a phone app via bluetooth, and have it send the data in real time so the graphs can be redrawn.
  * Assemble product, and show a full working stack. To show the full stack, we will
     * Show a series of prepicked commands can be sent to the hardware from the pi, and the data being observed can be sent to the phone.
     * Show an untrained AI on the pi can be integrated and send commands to the hardware.

    The idea of this is to show a fully working product, with the major functionality set up. The untrained AI is used in place while we gather data and perform the offline training. For functionality purposes, all we need to show is decisions being made, and user based reward-learning occuring. Further along, we can provide an AI which actually makes good decisions.

We are meeting up on Sunday (\textbf{2017-03-12}) to finish the integration of the above tasks, and possibly record video of the working prototype.

The simulator is finished, and I am able to run tests which show learning occuring by looking at the increase in system reward (with a higher reward equating to a better brew). Tomorrow or Monday evening, I'll prepare that into a short presentation to cleanly show the working Q-learner.

On the administrative side, we need to
* Review design documents now that the Teensy is not being used.
* Create the rough draft of the poster so we can turn that in.

\textbf{03-23-2017}

It appears that in the joy of the last day of classes, I forgot about the final update... Silly me.

Anyways, this week we had final meetings with Dale, our sponsor, and Frank. Through these, we demonstrated a completed prototype which has full communiction of information: General commands from the phone, specific commands from the AI, and hardware control down to the physical actuators and data from the sensors back to the AI and phone. It sure is cool to see it all working!

In the immediate future, the next steps are to create the final progress report for winter term and the group presentation of the project.
In the larger scale, we will attempt to create a larger, second prototype in the coming weeks which not only will have a larger brewing chamber, but will have room for additional sensors to connect to the Pi.

\textbf{2017-04-16}

Of note, the first half of this week I was in Canada visiting other schools.
This last week I met with Kirsten on Friday morning to discuss our poster, and got some good feedback. I'll be implementing that in our poster this evening.
I've also continued to arrange the weekly work times on Thursdays where we meet and discuss/work on the project.

In addition to this, I've been reworking the Python code on the RPi, to make it clean and better organized. This has been going well, as it's not hard to make Python look nice when you follow the language standards.

For this next week we will be finishing gathering data from the brewing device, as well as continuing to rework the current code/product and make it look nice for expo.

\textbf{2017-04-28}

This week we mainly worked on the poster and getting everything tied up cleanly. There are only a few minor edits to the code to upload to github this weekend, so things look easy on that end. There's not much news honestly, since things have wrapped up nicely here toward the end. After expo, the document writing should go smoothly, and pose liffle difficulty to finishing the course.

\textbf{2017-05-26}

This is the post-expo blog post, so I'll answer the questions in order.

1. If you were to redo the project from Fall term, what would you tell yourself?
Set aside an hour each day in the morning, when everything is quiet, to work on the code development. Development itself doesn't take that much time, so splitting it up each day will probably see the main development done by the midterm mark with ease.

2. What's the biggest skill you've learned?
Project management, for sure. Making sure the tea operates cleanly and gets all their parts done on time is not easy, and having taken on that roll I found myself challenged in a way I haven't really seen before.

3. What skills do you see yourself using in the future?
I do not see myself using many coding specific techniques or algorithms from this class again. I will probably use Q-learning again, but I've implemented that in the past. Project management would be the main skill where I will be continuing to use it again. I will have larger projects in the future, and keeping them on schedule will always be a non-trivial task.

4. What did you like about the project, and what did you not?
Technically, the project was very fun. I was able to work in Python, an language I am very comfortable with, and it had a cool concept. The most frustrating part of the project was the large ammount of writing associated with it. Writing for large projects is a given, but at many times it felt like there was no major reason behind the writing. Instead, writings in the latter two terms seemed tangential to actual project work.

5. What did you learn from your teammates?
Life gets in the way of people from time to time. It happens to all of us. Subsequently, I learned that when in a magerial role, I should plan for delays in others work just as I plan for them in mine.

6. If you were the client for this project, would you be satisfied with the work done?
I would say so. We have a working prototype of a product, which can be refined further with a little work. No project of significant scale can be developed in 6 months and be perfect.

7. If your project were to be continued next year, what do you think needs to be working on?
  1. Continued data collection
  2. Further physical refinement
  3. Creation of an end-to-end deplyment system so it works after just being plugged in

\subsection{Cody Holliday}

\textbf{2016-10-14}

This week went well. Our team is competent in writing as well as LaTeX, so the first couple writing assignments were quite quick. Our meeting with our sponsor helped us understand how we are going to incorporate business into the project. This gave us a little more structure in how to go about this project.

Hopefully next week we can begin our cycles of brewing so that eventually we can provide our algorithm with a god amount of data. Or alternatively work can begin on the device that will be making the mead. I have an arduino that we can use, so we are not held back by getting a microcontroller.

\textbf{2016-10-21}

This week was uneventful. We made revisions on our problem statement as well as read about the brewing process. Otherwise not much was accomplished. I was busy with other classes like CS 444.

\textbf{2016-10-30}

This week was stressful. Plans were made to brew mead, but unexpected surprises came up and we didn't end up going through with it. The requirements document took much longer than expected as not all of us split up the work evenly. Because of this we weren't able to have our document ready for our sponsor meeting, which is very unfortunate. Overall not enough time was spent doing the things we had to do.

\textbf{2016-11-04}

This week was challenging. Another group member and I had little time for the client requirements document, so the work on revision rested solely on one group member. This was stressful, as we ended up not finishing the document in time for our sponsor meeting. We weren't able to brew this week. It's incredible how much time you can spend doing things and yet not accomplish what you need to do. This sunday we are brewing mead as per our TA's request.

\textbf{2016-11-11}

This week we had a successful mead brewing session. Two jars of mead were made. Next week I hope we can ope them up to test their quality and measure our success. We did have some issues with the instructions we were given as they were vague on many of the measurements and times. Either we should use a different set of instructions, or we should request that our current instructions be updated.

\textbf{2016-11-18}

The past week has been hectic. We successfully completed our tech review and got a better idea of what we have to do for the project. I had issues with work ethic which lead to some increased workload, but we completed it. Next week we should be finished with our first batch of mead as well as have made a significant effort on our design document.

\textbf{2016-11-25}

Not much happened this week since about half of it was taken up by Thankgiving break. We're looking ahead to next week so that we can have the draft done much earlier than the due date. Hopefully we will have a draft done tuesday or wednesday. At least I think that's when we were thinking. It might be later than that. The batch is still undergoing the brewing process. We should probably go check it out.

\textbf{2016-12-2}

We successfully completed the preliminary design document, but not without some bumps. I had to rewrite the section on communication between the interface and the controller because my group mates disagreed with my choice. This was frustrating, but I was able to work through it. I hope next week we can finish the progress report in a timely manner.

\textbf{2016-1-27}

We were able to get two interviews in the past week, one from a coworker of mine and another from Connor's friend. In our meeting with our sponsor he was happy to have found that we conducted these interviews. I have not yet made progress on the bluetooth app communication, but in the coming week I believe that I will. In addition we will interview more people who are less experience with brewing.

\textbf{2016-2-18}

This last week was one of the least productive weeks I have had personally on this project. With all the events happening in this week from my birthday to Valentines Day, I was not able to bring my project up to alpha standards. However, I was able to lighten my load of work for the term so that I can focus on the project. Next week I pan on having at the least alpha functionality. That means two way Bluetooth and command protocol between the controller and the Android.

\textbf{2016-3-10}

This past week we outlined what we need to demonstrate for our weekly meeting. I was able to get large dataset communication as well as graphing working between the phone and my computer. I need to pair the phone with the raspberry pi before I can continue working. We are planning on having a separate Bluetooth listener thread that communicates with the main program through a FIFO as well as have continuous updates from the raspberry pi to the Android. I hope to get this completed this weekend.

\textbf{2017-3-21}

This last week was very productive. I created the layout for the brewing survey, which took a long time. Creating layouts for an app is like making a website, which I actually kind of despise. Learning how to best make a layout is kind of obnoxious as well. Other than that I added functionality to the survey so it sends a value back to the computer that represents what the user wrote in their survey. For the next week I hope to implement the raspberry pi telling the android that brewing has ended, resulting in the android prompting the user about the survey.

\textbf{2016-4-14}

This last week was good. We determined what we need to do for the rest of the term and what we want to have ready for expo. Since my portion is not as fleshed out as I would like it to be we will have to shoot for something that's a little less than planned. Next week I hope to have the user surveys done so that the AI can learn from user responses.

\textbf{2017-5-25}

This is the Post-Expo Blog!

1. If I had to redo the project starting fall term, I would tell myself to start on the bluetooth connectivity earlier. Getting that to work was such a hassle and it really kept me from doing more than I wanted to. In addition, I would say to work over winter brewk. It would have saved us much more hassle.

2.The biggest skill I learned was how to learn. This was the first time that I undertook a major project that had an effect on my grade. The pace at which I had to complete my work made me realize how little I usually had to do my homework. I learned how to be interested in learning and how to push myself to do real work.

3. I see myself using my research skills in the future. A lot of effort went into getting bluetooth to work as well as to make the app look good. In addition, some of the design techniques we used are certainly applicable to future projects.

4. I really liked some of the freedom we had. We created this project so we got to define what we wanted. Designing the app was really satisfying, but getting bluetooth connectivity was more than I could have wanted.

5. We worked pretty independently most of the time. I learned some interesting design methods as well as how to mesh communication between two different devices.

6.Well, initially I had a big mental image of what I wanted to make. Now that I have trudged through it all I would be sort of satisfied, but I would understand why it didn't live up to expectations.

7. I think what needs to be worked on is almost everything on my part. The way it acquires data and sends it to the android is completely independent of the main thread. There is nothing to catch disconnection, and if it does disconnect, the device goes into an undefined state where it can't communicate with the android and the main brewing thread is stuck at the end. This is troublesome since brews take months to make, and the user shouldn't be expected to stand there with the app open for months. The Android app is not fully implemented. It needs to query the raspberry pi about previous brews. It needs to have a settings section. The data it collects in the prebrewing state needs to actually be given to the AI to use. There are a lot of things that need to be improved about this project.

\textbf{2017-6-3}

Not much is left to be done for senior design. We completed the three writing assignments and now all we have to look forward to is the final report. I'm going to finish that by this Saturday since I will be on a trip during finals week in Bend.
