\begin{abstract}
	Home beer or mead brewing is not currently a process as simple as brewing coffee.
	We aim to change this, and make home brewing as easy as brewing coffee with the OpenBrew project.
	The OpenBrew project will consist of two major components:
	A hardware system for physically automating the brewing process, and a software system for managing the brewing process, utilizing machine learning to optimize mead or beer recipes.
	Through these components, a user will be able to easily incorporate a fully automated, intelligent home brewing system into an existing setup or as the framework to construct a new brewing system.
\end{abstract}

\newpage

\section{Definition of Problem}
It is not possible to brew mead or beer at home with the same ease as brewing a morning cup of coffee.
Previously, limitations on available technology prevented such a product from existing.
Thus brewing has remained limited to either professionals or dedicated hobbyists.
Today, more powerful, readily-available processors can be combined with easy-to-deploy networked sensors to create an easy to use product that can automate brewing.

However, hardware is not the only limit to easy to use, automated home brewing. 
Brewing a cup of coffee is a relatively simple and short process.
Fermentation is a more complex chemical process that takes place over a longer period of time.
This adds a huge amount of complexity to the problem, which contributes to the eclectic nature of home brewing.
By combining the readily available hardware with powerful algorithms from the fields of machine learning and artificial intelligence, we can create an effective automated home brewing setup.

The market for home brewing in the Pacific-Northwest, while robust, is a place full of potential.
Particularly with mead, there are no items on the market that can truly automate the brewing process for the common person, from start to finish.
This intersection of eclectic Pacific-Northwest culture and modern technology is where our project come in to fill the need.

\section{Proposed Solution}
The solution we propose is twofold in nature.
The first piece is a hardware implementation, which is controlled by the second piece, 
the autonomous software.

\subsection{Hardware}
We will construct our brewing device with common, off-the-shelf components
to create a simple and modular design. The reason for this decision is 
lower end hardware is more accessible to customers new to brewing.
Additionally, a microcontroller paired with general purpose sensors will be 
introduced into the brewing setup to measure and control the brewing process.
These components should not be cost prohibitive for the average home-brewer.

\subsection{Software}
The most important software aspect is the learned control policy 
to automate the brewing process. By providing a learning algorithm with the proper 
inputs at initialization and throughout the brewing process, the computer can completely 
control the complex chemical process. This allows the computer to improve the product 
based on the feedback from the user. To simplify this from an engineering and user 
perspective, we suggest a three part software solution.

The first part would be a simple interface for the user so they can 
interact with the control policy on a high level without needing to know concepts of 
machine learning. Secondly, there will need to be specific code for the microcontroller 
that can interface with a predetermined complex control policy. This interface would 
require the ability to translate inputs from the raw sensor data into a format for 
the policy to accept, and to take the  output of the policy and translate into specific 
actions. The third part is the machine learning and artificial intelligence aspect.
Here, we will need to construct a method of creating and training a control policy 
that can satisfy the following:

\begin{enumerate}
	\item Accept feedback from the user, and incorporate this feedback into the 
		policy for fine-tuning.
	\item Provide real-time control commands based on the continuous input that 
		can be used by the microcontroller.
	\item Does not require extensive amounts of data for bootstrapping the learning 
		of the policy.
\end{enumerate}

\subsection{Business Aspects}
Since this proposal is being supported by the Austin Entrepreneurship Program, our 
solution also needs to incorporate aspects of marketability. In order to shape our 
product into something people would be interested in, we will incorporate ideas 
from interviews with prospective customers into our design. We will also create 
business documents such as a lean canvas business model to help plan how this product 
will grow in the future.
\section{Performance Metrics}
The measurable outcome of our project is threefold; we create a device that 
autonomously brews mead, compile data on the progress of our algorithm 
over time, and begin developing a business around this device. 
The device will have a user friendly interface that shows metrics and allows
for basic user control over the brewing process The algorithm will determine
how the mead will be brewed then interact with the microcontroller to brew 
the mead. The microcontroller will control the mechanical parts according to
the output of the algorithm. Throughout the process of development we 
will record and graph data showing how the algorithm changes as it learns 
from input we provide. 

In addition to collecting data on batches of mead, we will be 
incorporating business practices into our project at every step. We will 
develop a business plan using the lean canvas model template, talk to 
brewers and customers about the brewing process, and incorporate at least 
three suggestions for our device in our final project.

The cost to build our final product should be reasonably priced for the new home brewer. 
The price of the device will be a relative proportion of the costs of home brewing kits.

