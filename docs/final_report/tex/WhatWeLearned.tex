\section{What Did We Learn?}
\subsection{Cody Holliday}
I think overall, I learned how to learn. This year was a struggle for me because I had to push myself to be an independent worker.
I have worked on an independent project before, but I was terrible at it.
I didn't know how to do anything, and I didn't have the drive to learn independently. 
I relied heavily on other people to help me with my work.
But going through Senior Design has really shown me how to manage my time and how to have the drive to work independently.
Working in teams was nothing new to me. However, I did learn how to effectively be part of a team.
That's not something that can be taught outside of a class like this, since most only last ten weeks.
The technical information I learned was not too different than what I learned before.
Sockets and basic programming were normal to me. What was different was enabling Bluetooth on the Pi and learning how activities worked on Android.
Throughout this year I have learned more and more about Linux and how kernel modules work.
Working with Bluetooth allowed me to put some of that knowledge to work.
In much the same way, learning about Android activities and fragments allowed me to use what I learned from my software engineering courses.
The Android activity is like a process. It's big, hefty, and has a lot of functionality.
Fragments are sort of like smaller activities. Their main purpose is to reuse parts of an Android activity to save memory and storage.
If I could do the whole project over again, I would use a framework when developing the app.
I wrote the app in java using Android studio, which is much more difficult than using a framework that does the heavy lifting for you.
That would save so much time and hassle that I would have had to deal with in the later stages of development.


