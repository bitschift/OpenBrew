\section{What Did We Learn?}
\subsection{Cody Holliday}
I think overall, I learned how to learn. This year was a struggle for me because I had to push myself to be an independent worker.
I have worked on an independent project before, but I was terrible at it.
I didn't know how to do anything, and I didn't have the drive to learn independently.
I relied heavily on other people to help me with my work.
But going through Senior Design has really shown me how to manage my time and how to have the drive to work independently.
Working in teams was nothing new to me. However, I did learn how to effectively be part of a team.
That's not something that can be taught outside of a class like this, since most only last ten weeks.
The technical information I learned was not too different than what I learned before.
Sockets and basic programming were normal to me. What was different was enabling Bluetooth on the Pi and learning how activities worked on Android.
Throughout this year I have learned more and more about Linux and how kernel modules work.
Working with Bluetooth allowed me to put some of that knowledge to work.
In much the same way, learning about Android activities and fragments allowed me to use what I learned from my software engineering courses.
The Android activity is like a process. It's big, hefty, and has a lot of functionality.
Fragments are sort of like smaller activities. Their main purpose is to reuse parts of an Android activity to save memory and storage.
If I could do the whole project over again, I would use a framework when developing the app.
I wrote the app in java using Android studio, which is much more difficult than using a framework that does the heavy lifting for you.
That would save so much time and hassle that I would have had to deal with in the later stages of development.

\subsection{Aravind Parasurama}
Most of my portion of the project was designing and building prototypes. This was an area I already had some familiarity with, however over this course
of this year, I learned plenty about designing hardware to very specific requirements. From writing a lot of software for the ATmega32u4 in the original design,
I also got a good refresher on microcontroller programming. As this project was automated brewing with AI, the team had to do some research on the mead and beer
brewing process, and as a result we all learned quite a bit on the subject. From speaking with people like the lead brewer of 10Barrel Brewing about my project,
I learned about the business potential a product like brew.ai could have in industrial brewing.

Group work is a challenge in communication. As long as everyone knows what's going on, and has a good idea of how what they're doing fits in to the project at scale,
a group project will go successfully. Projects in general require extensive planning, and reflective and iterative execution for best results. Looking back on the work
we did this year, one thing I would do differently would be sticking to a weekly meeting schedule where each team member gets some tasks to do each week. This system
will help keep all group members informed and never without a productive goal.

\subsection{Connor Yates}
What I truly learned this year was not a specific technology or coding related task. Rather, it was project management and scheduling. Our group suffered from missed goals and poor communication during the middle of the project. While we were all busy students, and it was understandable why we had reasons precipitating this scenario, I didn't want this to become the norm for the rest of the year. With some advice from Dale, I assumed the role of de facto project leader, making sure to organize and mobilize our team in an effective manner. While this effort was not perfect, I fully believe that it helped guarantee that our project was eventually successful. Having someone on the team dedicated to keeping the team on track with a schedule is imperative for overall success. I believe that if you want students to learn how to do project management for a long term project, have each student take on the role of project administrator one of the three terms. And during this, hopefully the student will realize how important project management is to success, and how much effort it can take.

I largely attribute my ability to learn some project management skills this year to finishing my coding section of the project with ease. My code was not difficult, and while I could have finished it sooner, I still finished before the end of winter term, and before it was needed by the rest of the project. This allowed me to allocate the extra time to scheduling, document revision, and technical help when requested. It was at this point where  our poor internal performance as a team began to turn around. We were having additional weekly meetings, and coordinating and communicating more efficiently.

If I were to do this project all over, I would start off as a stronger team leader. Teams are inherently more difficult to work with, since you have to coordinate actions between yourself and a host of others. This coordination could be done in a decentralized fashion, but I suspect that this would require a rare combination of students to successfully pull off. Instead, the coordination is better done as a centralized system, with one person dedicating a portion of their effort into being the coordinator for the team. This way, we would be efficient from the start, and would have a much better time with senior design as a whole.
