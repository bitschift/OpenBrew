\section{Tech Information Sources}
\subsection{Connor Yates}
Technologically, I did not learn much this year. The specific libraries, programming languages, and techniques I had all used before. Implementing my own deep Q network from scratch was new, but not something I was unfamiliar with. Subsequently, my experience learning about new technological information was minimal, and focused mainly on tips for implementing deep Q network. For these, I turned to people around me and to books by my side. Naturally, I used the official Python and Keras documentation extensivly, but this was used in a referential role, not as a means of learning.

To help with the implementation of the deep Q network, I mainly conversed with grad student friends, especially Will Curran, here at OSU who I knew were knowledgeable on the subject. Additionally, Steward Russell and Peter Norvigs' textbook ``Artificial Intelligence: A Modern Approach'' was a useful reference to have on hand while creating the learning agent. While it is the book used in the introductory courses on AI here at OSU, it still is a reliable reference for programming the basics of a learning agent.

\subsection{Cody Holliday}
There were only two sites that really helped me with this process: Stack Overflow and the Android documentation.
Finding information about implementing Bluetooth on a Raspberry Pi was the biggest hassle of the project.
I initially started using an HID Bluetooth dongle, so I looked through the man pages to find out how to use it, but ended up finding out that it had a kernel bug that caused it to malfunction.
I found information to enable normal HCI Bluetooth from a blog post about connecting an Android phone to a Raspberry Pi.
This blog post is on blog.davidvassallo.me.
The order of usefulness is Stack Overflow, Android Documentation, and blog.davidvassallo.me.
The blog is last since it solves a very specific problem, while the others helped with many other problems.
