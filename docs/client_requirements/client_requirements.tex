\documentclass[draftclsnofoot,onecolumn,letterpaper,10pt]{article}
\pagestyle{empty}
\usepackage{geometry}
\geometry{textheight=9.5in, textwidth=7in}

\author{Connor Yates\\
\texttt{yatesco@oregonstate.edu}
\and
Aravind Parasurama\\
\texttt{parasura@oregonstate.edu}
\and
Cody Holliday\\
\texttt{hollidac@oregonstate.edu}}
\date{\today}
\title{OpenBrew Problem Statement}
\begin{document}
\maketitle
%As a:
%I want:
%So that:
\section{Overview}
Section 2 describes the scope of the requirements for the BrewAI project.
Section 3 defines terms used in the document as well as terms that are
normally used when discussing the project. Section 4 outlines the project
in a 
\subsection{Scope}

\section{Definitions}
\begin{description}
	\item[Actuators] Electromechanical devices that allow the microprocessor to effect the fermentation process.
	\item[Artificial Intelligence] A learned control policy which controls the microcontroller.
	\item[AI] See Artificial Intelligence.
	\item[Brewing Setup] The collection of brewing vessels where fermentation occurs.
	\item[Microcontroller] A small, simple computer connected to sensors and actuators on the brewing setup.
	\item[Sensors] A collection of electronic sensors which measure useful data, such as temperature and specific gravity.
\end{description}

\section{Introduction}
Broadly, this document serves to layout the requirements of this project.
These requirements focus on both business and technological aspects.
This document aims to assist in
\begin{itemize}
	\item Providing specific deliverables and deadlines for the developers.
	\item Define the initial scope of the project.
	\item Create a structure to advance the business possibilities of the project.
	\item Incorporating the mindset and wants of the user into the final project.
\end{itemize}
To this end, the document provides an overview of the project, followed by the specific requirements.
The specific requirements will provide context for specific goals, and means by which to measure the attainment of the goals.

\section{Overall Description}


\section{Specific Requirements}
\subsection{Customer User Stories}
\begin{itemize}
	\item As a customer with no brewing experience, I want an interface that is simple and streamlined so that I don't have to deal with the technical details of brewing.
	\item As a user I want the data gathered by the device to be saved so that if I unplug the machine I won't lose my past brewing information.
	\item As a hobbyist brewer I want to be able to access advanced brewing settings so that I can precisely manipulate a brewing batch.
	\item As a user I want the device to be waterproof so that the batch doesn't destroy the computer inside.
	\item As a customer I want a brewing device that is easy to take apart so that it is easy to clean.
	\item As a user I want the device to detect when a batch has gone bad so that I am not harmed by incorrectly made batches.
	\item As an amateur brewer I want the AI to control the brewing hardware so that I don't have to operate the hardware myself.
	\item As a user I want to be able to select what kind of drink I want so that I can have a better variety of drinks that it can make.
	\item As an amateur brewer I want a device that will learn how to make better batches so that I can drink tasty alcoholic beverages.
	\item As a user I want the device to tell me if my inputs would make a bad batch so that I don't waste ingredients.
\end{itemize}
\subsection{Technical Requirements}
Based upon our own ideas of for the project, as well as the concepts from the user stories, the technical requirements can be summarized as follows.
These requirements are presented in order of their completion.
I.e., requirements must be completed in their presented order.
\begin{itemize}
	\item Assemble a brewing setup.
	\item Acquire electronics hardware - microcontroller, actuators, and sensors.
	\item Gather data for training machine learning algorithm.
	\item Design learning structure for the control policy.
	\item Create the learning structure and control policy.
	\item Train the policy.
	\item Use the policy to automate the fermentation process.
	\item Create a user interface to monitor and control the brewing process.
\end{itemize}
\end{document}
