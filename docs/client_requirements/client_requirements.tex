\documentclass[letterpaper,10pt]{article}
\pagestyle{empty}
\usepackage{geometry}
\geometry{textheight=9.5in, textwidth=7in}

\author{Connor Yates\\
\texttt{yatesco@oregonstate.edu}
\and
Aravind Parasurama\\
\texttt{parasura@oregonstate.edu}
\and
Cody Holliday\\
\texttt{hollidac@oregonstate.edu}}
\date{\today}
\title{OpenBrew Problem Statement}
\begin{document}
\maketitle

\newpage

\section{Overview}
Section 2 describes the scope of the requirements for the BrewAI project.
Section 3 defines terms used in the document as well as terms that are normally used when discussing the project.
Section 4 outlines the project in a high level, whereas Section 5 describes the requirements through user stories.

\subsection{Scope}
BrewAI is a device that automatically brews mead or beer.
The scope of this document focuses on the specifications of the device as well as the steps to build a business around it.


\section{Definitions}

\section{Introduction}
\section{Overall Description}
BrewAI is a device that can automatically brew as well as learn how to improve on recipies over time.
The device will have a streamlined and simple interface that will interact with the AI.
The interface will also have more advanced settings for hobbiest brewers.
The AI will learn using reenforcement learning, then operate the hardware through a microcontroller.
The hardware will be waterproof and modular for easy cleaning.

\section{Specific Requirements}
As a customer with no brewing experience\\
I want an interface that is simple and streamlined\\
So that I don't have to deal with the technical details of brewing.\\

As a customer\\
I want a brewing device that is easy to take apart\\
So that it is easy to clean\\

As an amateur brewer\\
I want a device that will learn how to make better batches\\
So that I can drink tasty alcoholic beverages.\\

As a hobbyist brewer\\
I want to be able to access advanced brewing settings\\
So that I can precisely manipulate a brewing batch.\\
\\
As an amateur brewer\\
I want the AI to control the brewing hardware\\
So that I don't have to operate the hardware myself\\

As a user\\
I want the data gathered by the device to be saved\\
So that if I unplug the machine I won't lose my past brewing information.\\

As a user\\
I want the device to be waterproof\\
So that the batch doesn't destroy the computer inside.\\

As a user\\
I want the device to detect when a batch has gone bad\\
So that I am not harmed by incorrectly made batches.\\

As a user\\
I want to be able to select what kind of drink I want\\
So that I can have a better variety of drinks that it can make.\\

As a user\\
I want the device to tell me if my inputs would make a bad batch\\
So that I don't waste ingredients.\\

\subsection{Business Requirements}
As we develop this project we must treat it as if we are starting a business.
To go about starting a business we have to go through a series of steps to set it up.
\begin{itemize}
	\item Develop a business plan using the Lean Canvas Model
	\item Interview businesses and homebrewers about their needs as brewers as well as what they want to change about their process.
	\item After developing a prototype, ask businesses and homebrewers about possible features to add to the device.
\end{itemize}

\end{document}
